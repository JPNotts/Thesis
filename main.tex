\documentclass[oneside, 12 pt]{book}

\usepackage{amsmath, amssymb}

\usepackage[english]{babel}
\usepackage{microtype} %Tidyness
\usepackage{dsfont} %math 1
\usepackage[left = 4cm, right = 2.5cm, top = 2.5cm, bottom = 2.5cm]{geometry} %% Margin width
\linespread{1.5}
\usepackage{subfig} %% side by side figures
\usepackage{pdflscape} %%landscape pages
\usepackage{multirow} %% Multi Rows in tables
\usepackage{hyperref} %hyperlink table of contents
%\usepackage{algorithm}
%\usepackage{algorithmicx}
\usepackage{algpseudocode}
\usepackage{textcomp}
\usepackage{eurosym} %eurosymbol
\usepackage{graphicx}
\setcounter{secnumdepth}{4} %numbering for subsubsection 
\usepackage[Sonny]{fncychap} %Chapter Titles
\usepackage{afterpage} %spacing before landscape
\usepackage[version=3]{mhchem} % For Ca^2+
\let\oldemptyset\emptyset % empty set symbol
\usepackage{bbm} % for \mathbbm{}
\usepackage{bm}
\usepackage{mathtools} % for \splitdfrac{}

% Colours
\usepackage[dvipsnames]{xcolor}
\definecolor{col1}{RGB}{102, 194, 165}
\definecolor{col2}{RGB}{252, 141, 98}
\definecolor{col3}{RGB}{141, 160, 203}
\definecolor{col4}{RGB}{231, 138, 195}
\definecolor{col5}{RGB}{166, 216, 84}
\usepackage{colortbl} %colour table
% Algorithms 
%\usepackage{algorithm}
\usepackage[ruled,vlined]{algorithm2e}
%\usepackage[noend]{algpseudocode}
\makeatletter
\def\BState{\State\hskip-\ALG@thistlm}
\makeatother
%\usepackage[]{algorithm2e}

% Formatting for steps.
\usepackage{enumitem}
\newlist{steps}{enumerate}{1}
\setlist[steps, 1]{label = \it Step \arabic*:}

%Headers and Footers
\usepackage{fancyhdr}
\setlength{\headheight}{15pt}

\pagestyle{fancy}
\renewcommand{\chaptermark}[1]{ \markboth{#1}{} }
\renewcommand{\sectionmark}[1]{ \markright{#1} }

\fancyhf{}
\fancyhead[LE,RO]{\thepage}
\fancyhead[RE]{\ifnum \value{chapter}>0  Chapter \nouppercase{\thechapter : \leftmark} \fi} 
\fancyhead[LO]{ \ifnum \value{chapter}>0 Chapter \nouppercase{\thechapter : \leftmark} \fi } 

\fancypagestyle{plain}{ %
  \fancyhf{} % remove everything
  \renewcommand{\headrulewidth}{0pt} % remove lines as well
  \renewcommand{\footrulewidth}{0pt}
}
\AtBeginDocument{%
  \addtocontents{toc}{\protect\thispagestyle{empty}}%
  \addtocontents{lof}{\protect\thispagestyle{empty}}%
}


\usepackage{url}
\hypersetup{
    colorlinks,
    citecolor=black,
    filecolor=black,
    linkcolor=black,
    urlcolor=black
}
%Tikz Library
\usepackage{tikz}
\usetikzlibrary{positioning, calc}
\usepackage[round]{natbib}
\newcommand\StateX{\Statex\hspace{\algorithmicindent}}

\newcommand{\lc}{\mathcal{l}}
\newcommand{\etal}[0]{\textit{et al.} }
\newcommand{\betaij}[0]{\beta_{i, j}}
\newcommand{\betajk}[0]{\beta_{j, k}}
\newcommand{\boldbeta}[0]{\boldsymbol{\beta}}
\newcommand{\GP}[0]{\mathcal{GP}(0, \> \Sigma)}
\newcommand{\f}[0]{f}
\setlength{\marginparwidth}{1.9cm}
\usepackage{subfiles}
\graphicspath{{figures/}{../figures/}}
%\DeclareUnicodeCharacter{2212}{-}
 
\usepackage{blindtext}
 
\begin{document}
 
 \thispagestyle{empty}
\begin{center}
{\large\textsc{University of Nottingham}}\\

\begin{figure}[!h]
\centering
\includegraphics[width=4.5cm]{UoNlogo}
\end{figure}

{\vspace*{0.7cm}}

{\large\textsc{School of Mathematical Sciences}}\\

\vspace{\stretch{1}}

{\huge{\bfseries{Cell Signalling}}\par}
\vspace*{2cm}
{\Large{Jake Powell}}\\

\vspace{\stretch{1}}

{\large A thesis submitted to the University of Nottingham for the degree of \par}
{\large \textsc{Doctor of Philosophy} \par}

\vspace{\stretch{1}}
{\large{\textsc{One Day}}}\pagebreak
\end{center}


 
\frontmatter

\addcontentsline{toc}{chapter}{Abstract}%
\chapter*{Abstract}
Original abstract: Intracellular Calcium oscillations is a versatile signalling mechanism responsible for many biological phenomena including immune responses and insulin secretion. There is now compelling evidence that  whole-cell calcium oscillations are stochastic, arising from random molecular interactions at the subcellular level. This poses a significant challenge for modelling. Here, we utilise a probabilistic method that treats calcium oscillations as a point process. By employing an intensity function, we capture intrinsic cellular heterogeneity as well as inhomogeneous extracellular conditions, such as time-dependent stimulation. I will demonstrate how to simulate stochastic calcium spikes based on intensity  functions. Furthermore, we will infer model parameters from real data using Bayesian methods, and utilise novel MCMC techniques to deal with priors on functions. 
 
\newpage

%\noindent{\Large \textbf{Relevant Publications}}
%\begin{itemize}
%	\item  Chapter 2 and Section 5.3 is contained in Seymour, R. G., Kypraios, T., O'Neill, P. D., and Hagenaars, T. J. (2020). A Bayesian Nonparametric Analysis of the 2003 Outbreak of Avian Influenza in the Netherlands. \textit{J. R. Stat. Soc. C}. 
%	\item Chapter 3 and Section 5.4 is contained in Seymour, R. G., Kypraios, T., O'Neill, P. D. O. (2020). Bayesian Nonparametric Methods for Individual-Level Stochastic Epidemic Models. \textit{J. R. Stat. Soc. B}. 
%	%\item Chapter 2 and Section 5.3 is being prepared for publication.
%	%\item Chapter 2 and Section 5.4 is being prepared for publication.
%\end{itemize}

 \vspace{3cm}
 
 
\noindent{\Large \textbf{Acknowledgements}} \\ \\
Thank some folk.

%\pagestyle{empty}
\tableofcontents
\listoffigures
\listoftables
\mainmatter
\pagestyle{fancy}
\chapter{Introduction}
\subfile{Chapter1/Chapter1} 
 
\chapter{Model and Bayesian Inference}\label{chapter 2}
 
\subfile{Chapter2/Chapter2}

\chapter{Non-trivial Model Properties}\label{chapter 3}

\subfile{Chapter3/Chapter3}

\chapter{Refractory Period}\label{chapter 4}

\subfile{Chapter4/Chapter4}

\chapter{Application to Real Data}\label{chapter 5}

\subfile{Chapter5/Chapter5}

\chapter{Clustering Intensity Functions}\label{chapter 6}

\subfile{Chapter6/Chapter6.tex}

\chapter{Conclusion}\label{chapter 7}

\subfile{Chapter7/Chapter7.tex}



\bibliographystyle{apalike}
\bibliography{bibliography} 
 
\end{document}  