% Chapter 3.
\documentclass[../main.tex]{subfiles}
\begin{document}

\subsection{How it affects the model}
We will consider having a Gamma ISI throughout, although the same concept has been used for Poisson, inverse Gaussian and Weibull. Previously, without the inclusion of $T_{\mathrm{min}}$ we have the following ISI distribution.  

 \begin{equation}
p(y_i,y_{i-1}|x) = \frac{\gamma x(y_i)}{\Gamma (\gamma)} \big[ \gamma X(y_{i-1},y_i) \big]^{\gamma - 1} \mathrm{e}^{-\gamma X(y_{i-1},y_i) },
\end{equation}
 where
 \begin{equation}
X(s,t) = \int^{t}_{s} x(u) du , \qquad \text{for any } s,t \in [0,T],
\end{equation}
where $\gamma > 0 $ is the ISI parameter and $\Gamma$ is the Gamma function.\\
 
 Now with the inclusion of $T_{\mathrm{min}}$ the ISI distribution changes to 
  \begin{equation}
p(y_i,y_{i-1}|x) =
\begin{cases}
0  & \text{ if } y_i-y_{i-1}<  T_{\mathrm{min}} \\
 \frac{\gamma x(y_i)}{\Gamma (\gamma)} \big[ \gamma X(y_{i-1},y_i) \big]^{\gamma - 1} \mathrm{e}^{-\gamma X(y_{i-1},y_i) } & \text{o/w}\\
 \end{cases}
\end{equation}

However, we can rewite this equation and alter the definition of $X$ to account for the refractory period. Thus,
\begin{equation}
p(y_i,y_{i-1}|x) = \frac{\gamma x(y_i)}{\Gamma (\gamma)} \big[ \gamma \tilde X(y_{i-1},y_i) \big]^{\gamma - 1} \mathrm{e}^{-\gamma \tilde X(y_{i-1},y_i) },
\end{equation}
 where
 \begin{equation}
\tilde X(s,t) = \int^{t}_{s+T_{\mathrm{min}}} x(u) du , \qquad \text{for any } s,t \in [0,T],
\end{equation}

Next we want to calculate the likelihood for an entire spike sequence. In general we have: 
\begin{equation}
p(\mathbf{y} |x) = p_1(y_1|x) p_T(T,y_N|x) \prod^N_{i=2} p(y_i, y_{i-1}|x).
\end{equation} 

Above we have dealt with the ISI distribution, however we need to decide how $T_{\mathrm{min}}$ affects $p_1$ and $p_T$. We assume that $T_{\mathrm{min}}$ has no effect on $p_1$, since we do not know the time of the last spike before the initial spike. Hence, the refractory period may already of occurred.
Thus, 
\begin{equation}
p_1(y_1|x) = x(y_1)e^{-X(0,y_1)}  
\end{equation}

$T_{\mathrm{min}}$ however does effect $p_T$. If the time difference between the last spike and end time is less then the refractory period then $p_T$ can be disregarded as it has no effect on the likelihood. If the difference is longer then the refractory period then we have 
\begin{equation}
p_T(T,y_N|x) = e^{-\tilde X(y_N,T)}, 
\end{equation}

Thus when calculating the likelihood for $x$ we do not need to change the formula, we only need to change our calculation of $X$ by including $T_{\mathrm{min}}$ to the bottom of the integrand. However we will need to distinguish the case whether to include a term for $p_T$ or not. 

\subsection{Estimating $T_{\mathrm{min}}$}
We estimate $T_{\mathrm{min}}$ using MCMC, where if we are estimating other parameters ($x(t)$/ISI parameters) we use Gibbs sampling and sample from $T_{\mathrm{min}}$ separately.  

$T_{\mathrm{min}}$ only affects the likelihood via $X(y_{i-1},y_i)$ hence we get the
marginal for each ISI distribution reduces to : \\


\begin{tabular}{ll}
Distribution & log marginal \\ \hline
Poisson & $-X(y_{i-1},y_i)$ \\
Gamma & $(\gamma - 1) \log X(y_{i-1},y_i) - \gamma X(y_{i-1},y_i)$ \\
Inverse Gaussian & 1 \\ 
Log Normal & 1 \\
Weibull  & \\
\end{tabular} 

\end{document}